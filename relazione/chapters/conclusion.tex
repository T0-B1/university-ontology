\chapter*{Conclusioni}

A nostro avviso, la rete delle ontologie e dei vocabolari controllati che compongono OntoPiA rappresenta un primo e importante passo verso la creazione di un vero e propio "grafo della conoscenza" della pubblica amministrazione italiana.

OntoPiA getta le basi per creare un modello formale che permetta la standardizzazione, l'interoperabilità e, in ultima istanza, l'apertura del patrimonio dati del paese.

Sarà interessante monitorarne l'evoluzione nel futuro prossimo per vedere quali enti, pubblici o privati, prenderanno parte all'iniziativa così come sarebbe interessante partecipare attivamente al progetto contribuendo con lo sviluppo di un'ontologia, cosa che verrà valutata.

\addcontentsline{toc}{chapter}{Conclusioni}