\chapter{Esperimenti codice}

Esperimenti latex da eliminare quando si consegnerà, ci ricoderemo? B)

 \colorbox{lightgray}{sdfASDF}

 \colorbox{Gainsboro}{sdfASDF}

 \colorbox{Gainsboro!60!Lavender}{sdfASDF}

 \colorbox{WhiteSmoke}{sdfASDF} 

 \colorbox{WhiteSmoke!70!Lavender}{sdfASDF}

 \begin{quote}
    \centering
     prova
 \end{quote}

 \begin{table}
     \centering
     \begin{tabular}{c|c}
     \hline
         prova1 & prova2 \\
         prova3 & prova4
     \end{tabular}
     \caption{Caption}
     %\label{tab:AAAAel}
 \end{table}

 \textcolor{red}{METTERE IN GRASSETO I NOMI DELLE CLASSI?}
\begin{table}[h]
     \centering
     \begin{tabular}{c c}
     \textbf{Classe} & \textbf{Definizione} \\
     \hline
         Car Park & Classe principale dell'ontologia rappresenta il concetto stesso di parcheggio. \\
         Car Park Typology & Rappresenta la tipologia del parcheggio. Le istanze di tale classe sono definite dal relativo vocabolario controllato. Principalmente possono essere suddivisi in parcheggi a sviluppo verticale (parcheggi interrati, in elevazione, misti) e parcheggi a sviluppo orizzontale (parcheggi in superficie e parcheggi a raso o a livello). \\
         Car Sharing Car Park & ... \\
         offer & ... \\
         online contact point  & ... \\
         Park and Ride & ... \\
         point of interest & ... \\
         Rotating Car Park & ... \\
         Terminal Car Park & ... \\
         topic & ...
     \end{tabular}
     \caption{Classi usate nell'ontologia PARK.}
     %\label{tab:my_label}
 \end{table}

\begin{tabular}{l c p{.5\textwidth}}

\toprule
$R$ is...       &                  & ...if and only if... \\
\midrule
irreflexive     &$:\Leftrightarrow$& for all $x$ iht $xx \notin R$. \\
transitive      &$:\Leftrightarrow$& for all $xyo$ iht if $xo \in R$ and $oy \in R$, then $xy \in R$. \\
antisymmetric   &$:\Leftrightarrow$& for all $xy$ iht if $xy \in R$ and $yx \in R$, then $x=y$. \\
a partial order &$:\Leftrightarrow$& $R$ is transitive and antisymmetric. \\
\bottomrule
\end{tabular}



\begin{xltabular}{\linewidth}{ l  X }
    \caption{Description of Variables used in this Study} 
    %\label{table: vardescription}\\
    \endlastfoot 
    \hline
    \textbf{\normalsize Code} & \textbf{\normalsize Definition and source}  \\
    \hline
    
    \endhead
    
    \textbf{exportsgr} & Exports of goods and services (annual \% growth) retrieved from World Bank. \\ \hline 
    
    \textbf{importsgr} & Imports of goods and services (annual \% growth) retrieved from World Bank.\\ \hline 
    
    \textbf{importsgr} & Imports of goods and services (annual \% growth) retrieved from World Bank.\\ \hline 
    
    \textbf{gr\_tot} & Terms of trade change over previous year (in \%).  Data for terms of trade are collected from theglobaleconomy.com  and Kaminsky and Reinhart online database. Since variables have two different base years, the base year for both was changed to 2000 to have the same base year. And then the change is calculated as below. \\
    \hline

\end{xltabular}
